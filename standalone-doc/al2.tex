
% al2.tex - Colin Ramsay - 2 Feb 01
%
% The AL2 chapter.
%
%   5   10   15   20   25   30   35   40   45   50   55   60   65   70   75
% ..|....|....|....|....|....|....|....|....|....|....|....|....|....|....|

Level 2 of \ace\ is a complete, standalone application for generating and
  manipulating coset tables.
It can be used interactively, or can take its input from a script file.
It is reasonably robust and comprehensive, but no attempt has been made to
  make it `industrial strength' or to give it any of the features of, say,
  \textsc{Magma} \cite{BCP} or \textsf{GAP} \cite{Sch}.
Most of its features have been added in response to user requests, and it
  is assumed that the user is `competent'\kern-1.5pt.
One of the primary goals in developing \ace\ was to demonstrate how to
  correctly use \ace\ Levels $0$ \amp $1$; some care is taken to ensure
  that the user cannot generate `invalid' tables.

A complete description of all the Level 2 commands is given in
  Appendix~\ref{app:cmd}.

\section{Enumeration mode \amp style}\label{sec:style}

The core enumerator takes two arguments, which select the enumeration
  mode and style.
The mode determines whether or not we retain any existing table
  information.
Initially, we start with an empty table and use the begin mode (the
  \ttt{beg} command).
This can be followed by a series of continue and/or redo modes (the 
  \ttt{cont} \amp \ttt{redo} commands) which build on or modify the
  table generated by the begin mode.
So it is possible to do an enumeration in stages, altering the parameters
  at each stage.
Various interlocks are present to prevent a combination of choices which
  (potentially) leads to an invalid table.

The enumeration style is the balance between C-style definitions (i.e.,
  coset table based, Felsch style) and R-style definitions (i.e., relator
  based, HLT style), and is controlled by the \ttt{ct} \amp \ttt{rt}
  parameters.
The absolute values of these parameters sets the number of definitions
  (C-style) or coset applications (R-style) per pass through the
  enumerator's main loop.
The sign of these parameters sets the style, and the possible combinations
  are given in Table~\ref{tab:sty}

\begin{table}
\hrule
\caption{The styles}
\label{tab:sty}
\smallskip
\renewcommand{\arraystretch}{0.875}
\begin{tabular*}{\textwidth}{@{\extracolsep{\fill}}crrlc} 
\hline\hline
& \ttt{Rt} value & \ttt{Ct} value & style name & \\
\hline
& $<\!0$ & $<\!0$ & R/C & \\
& $<\!0$ & $0$    & R*  & \\
& $<\!0$ & $>\!0$ & Cr  & \\
& $0$    & $<\!0$ & C   & \\
& $0$    & $0$    & R/C (defaulted) & \\
& $0$    & $>\!0$ & C  & \\
& $>\!0$ & $<\!0$ & Rc & \\
& $>\!0$ & $0$    & R  & \\
& $>\!0$ & $>\!0$ & CR & \\
\hline\hline
\end{tabular*}
\end{table}

In R style all the definitions are made via relator scans; i.e., this
  is HLT mode.
In C style all the definitions are made in the next empty table slot and
  are tested in all essentially different positions in the relators; i.e.,
  this is Felsch mode.
In R/C style we run in R style until an overflow, perform a lookahead on
  the entire table, and then switch to CR style.
Defaulted R/C style is the default style, and here we use R/C style with
  \ttt{ct:1000} and \ttt{rt} set to approximately $2000$ divided by the
  total length of the relators, in an attempt to balance R \amp C
  definitions when we switch to CR style.
Rc \amp Cr styles are like R \amp C styles, except that a single C or R
  style pass (respectively) is done after the initial R or C style pass.
R* style makes definitions the same as R style, but tests all definitions
  as for C style.
In CR style alternate passes of C style and R style are performed, with
  all definitions tested.
The $\ttt{Ct}<0$ C style is reserved for future use, and should not be
  used.

\section{Predefined strategies}\label{sec:strat}

The versatility of \ace\ means that it can be difficult to select
  appropriate parameters when presented with a new enumeration.
The problem is compounded by the fact that no generally applicable rules
  exist to predict, given a presentation, which parameter settings are
  `good'\kern-1.5pt.
To help overcome this problem, \ace\ contains various commands which
  select particular enumeration strategies.
One or other of these strategies may work and, if not, the results may
  indicate how the parameters can be varied to obtain a successful
  enumeration.
The thirteen standard strategies are listed in Table~\ref{tab:pred}.

Note that we explicitly (re)set all of the listed enumerator parameters in
  all of the predefined strategies, even although some of them have no
  effect.
For example, the \ttt{fi} value is irrelevant in HLT mode.
The idea behind this is that, if you later change some parameters
  individually, then the enumeration retains the `flavour' of the last
  selected predefined strategy.
Note also that other parameters which may effect an enumeration are left
  untouched by setting one of the predefined strategies; for example, the
  values of \ttt{max} \amp \ttt{asis}.
These parameters have an effect which is independent of the selected 
  strategy.

\begin{table}
\hrule
\caption{The predefined strategies}
\label{tab:pred}
\smallskip
\renewcommand{\arraystretch}{0.875}
\begin{tabular*}{\textwidth}{@{\extracolsep{\fill}}lrrrrrrrrrrrrr} 
\hline\hline
          & \multicolumn{13}{c}{parameter} \\ 
\cline{2-14}
strategy & path & row & mend & no & look & com & ct   & rt    & fi & pmod & psiz & dmod & dsiz \\ 
\hline
def    & n   & y   & n    & -1 & n    & 10  & 0    & 0     & 0  & 3    & 256  & 4    & 1000 \\
easy   & n   & y   & n    & 0  & n    & 100 & 0    & 1000  & 1  & 0    & 256  & 0    & 1000 \\
fel:0  & n   & n   & n    & 0  & n    & 10  & 1000 & 0     & 1  & 0    & 256  & 4    & 1000 \\
fel:1  & n   & n   & n    & -1 & n    & 10  & 1000 & 0     & 0  & 3    & 256  & 4    & 1000 \\
hard   & n   & y   & n    & -1 & n    & 10  & 1000 & 1     & 0  & 3    & 256  & 4    & 1000 \\
hlt    & n   & y   & n    & 0  & 1    & 10  & 0    & 1000  & 1  & 0    & 256  & 0    & 1000 \\
pure c & n   & n   & n    & 0  & n    & 100 & 1000 & 0     & 1  & 0    & 256  & 4    & 1000 \\
pure r & n   & n   & n    & 0  & n    & 100 & 0    & 1000  & 1  & 0    & 256  & 0    & 1000 \\
sims:1 & n   & y   & n    & 0  & n    & 10  & 0    & 1000  & 1  & 0    & 256  & 0    & 1000 \\
sims:3 & n   & y   & n    & 0  & n    & 10  & 0    & -1000 & 1  & 0    & 256  & 4    & 1000 \\
sims:5 & n   & y   & y    & 0  & n    & 10  & 0    & 1000  & 1  & 0    & 256  & 0    & 1000 \\
sims:7 & n   & y   & y    & 0  & n    & 10  & 0    & -1000 & 1  & 0    & 256  & 4    & 1000 \\
sims:9 & n   & n   & n    & 0  & n    & 10  & 1000 & 0     & 1  & 0    & 256  & 4    & 1000 \\
\hline\hline
\end{tabular*}
\end{table}

Note that, apart from the \ttt{fel:0} \amp \ttt{sims:9} strategies, all of
  the strategies are distinct, although some are very similar.
Further details of each strategy are contained in their entry in
  Appendix~\ref{app:cmd}.
