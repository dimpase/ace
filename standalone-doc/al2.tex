%%%%%%%%%%%%%%%%%%%%%%%%%%%%%%%%%%%%%%%%%%%%%%%%%%%%%%%%%%%%%%%%%%%%%%%%%
%%
%W  al2.tex            ACE standalone documentation          Colin Ramsay
%W                                                            Greg Gamble
%%
%H  $Id$
%%

%%  Colin Ramsay - 21 Sep 1999  is the true author of this document
%%  Greg Gamble  - 28-29 Feb 2000 made minor modifications:
%%                * modifications detailed in CHANGES
%%                * put this header at the top for CVS

%%  The AL2 chapter.
%%
%%  5   10   15   20   25   30   35   40   45   50   55   60   65   70   75
%%..|....|....|....|....|....|....|....|....|....|....|....|....|....|....|

Level 2 of \ace\ is a complete, standalone application for generating and
  manipulating coset tables.
It can be used interactively, or can take its input from a script file.
It is reasonably robust and comprehensive, but no attempt has been made to
  make it `industrial strength' or to give it any of the features of, say,
  \textsc{Magma} \cite{BCP} or \textsf{GAP} \cite{Sch}.
Most of its features have been added in response to user requests, and it
  is assumed that the user is `competent'\kern-1.5pt.
One of the primary goals in developing \ace\ was to demonstrate how to
  correctly use \ace\ Levels $0$ \amp $1$; some care is taken to ensure
  that the user cannot generate `invalid' tables.

\section{Enumeration mode \amp style}\label{sec:style}

The core enumerator takes two arguments, which select the enumeration
  mode and style.
The mode determines whether or not we retain any existing table
  information.
Initially, we start with an empty table and use the begin mode (the
  \ttt{beg} command).
This can be followed by a series of continue and/or redo modes (the 
  \ttt{cont} \amp \ttt{redo} commands) which build on or modify the
  table generated by the begin mode.
So it is possible to do an enumeration in stages, altering the parameters
  at each stage.
Various interlocks are present to prevent a combination of choices which
  (potentially) leads to an invalid table.

The enumeration style is the balance between C-style definitions (i.e.,
  coset table based, Felsch style) and R-style definitions (i.e., relator
  based, HLT style), and is controlled by the \ttt{ct} \amp \ttt{rt}
  parameters.
The absolute values of these parameters sets the number of definitions
  (C-style) or coset applications (R-style) per pass through the
  enumerator's main loop.
The sign of these parameters sets the style, and the possible combinations
  are given in Table~\ref{tab:sty}

\begin{table}
\hrule
\caption{The styles}
\label{tab:sty}
\smallskip
\renewcommand{\arraystretch}{0.875}
\begin{tabular*}{\textwidth}{@{\extracolsep{\fill}}crrlc} 
\hline\hline
& \ttt{Rt} value & \ttt{Ct} value & style name & \\
\hline
& $<\!0$ & $<\!0$ & R/C & \\
& $<\!0$ & $0$    & R*  & \\
& $<\!0$ & $>\!0$ & Cr  & \\
& $0$    & $<\!0$ & C   & \\
& $0$    & $0$    & R/C (defaulted) & \\
& $0$    & $>\!0$ & C  & \\
& $>\!0$ & $<\!0$ & Rc & \\
& $>\!0$ & $0$    & R  & \\
& $>\!0$ & $>\!0$ & CR & \\
\hline\hline
\end{tabular*}
\end{table}

In R style all the definitions are made via relator scans; i.e., this
  is HLT mode.
In C style all the definitions are made in the next empty table slot and
  are tested in all essentially different positions in the relators; i.e.,
  this is Felsch mode.
In R/C style we run in R style until an overflow, perform a lookahead on
  the entire table, and then switch to CR style.
Defaulted R/C style is the default style, and here we use R/C style with
  \ttt{ct:1000} and \ttt{rt} set to approximately $2000$ divided by the
  total length of the relators, in an attempt to balance R \amp C
  definitions when we switch to CR style.
Rc \amp Cr styles are like R \amp C styles, except that a single C or R
  style pass (respectively) is done after the initial R or C style pass.
R* style makes definitions the same as R style, but tests all definitions
  as for C style.
In CR style alternate passes of C style and R style are performed, with
  all definitions tested.
The $\ttt{Ct}<0$ C style is reserved for future use, and should not be
  used.

\section{Predefined strategies}\label{sec:strat}

The versatility of \ace\ means that it can be difficult to select
  appropriate parameters when presented with a new enumeration.
The problem is compounded by the fact that no generally applicable rules
  exist to predict, given a presentation, which parameter settings are
  `good'\kern-1.5pt.
To help overcome this problem, \ace\ contains various commands which
  select particular enumeration strategies.
One or other of these strategies may work and, if not, the results may
  indicate how the parameters can be varied to obtain a successful
  enumeration.
The thirteen standard strategies are listed in Table~\ref{tab:pred}.

Note that we explicitly (re)set all of the listed enumerator parameters in
  all of the predefined strategies, even though some of them have no
  effect.
For example, the \ttt{fi} value is irrelevant in HLT mode.
The idea behind this is that, if you later change some parameters
  individually, then the enumeration retains the `flavour' of the last
  selected predefined strategy.
Note also that other parameters which may effect an enumeration are left
  untouched by setting one of the predefined strategies; for example, the
  values of \ttt{max} \amp \ttt{asis}.
These parameters have an effect which is independant of the selected 
  strategy.

\begin{table}
\hrule
\caption{The predefined strategies}
\label{tab:pred}
\smallskip
\renewcommand{\arraystretch}{0.875}
\begin{tabular*}{\textwidth}{@{\extracolsep{\fill}}lrrrrrrrrrrrrr} 
\hline\hline
          & \multicolumn{13}{c}{parameter} \\ 
\cline{2-14}
strategy & path & row & mend & no & look & com & ct   & rt    & fi & pmod & psiz & dmod & dsiz \\ 
\hline
default& 0   & 1   & 0    & -1 & 0    & 10  & 0    & 0     & 0  & 3    & 256  & 4    & 1000 \\
easy   & 0   & 1   & 0    & 0  & 0    & 100 & 0    & 1000  & 1  & 0    & 256  & 0    & 1000 \\
felsch:0& 0   & 0   & 0    & 0  & 0    & 10  & 1000 & 0     & 1  & 0    & 256  & 4    & 1000 \\
felsch:1& 0   & 0   & 0    & -1 & 0    & 10  & 1000 & 0     & 0  & 3    & 256  & 4    & 1000 \\
hard   & 0   & 1   & 0    & -1 & 0    & 10  & 1000 & 1     & 0  & 3    & 256  & 4    & 1000 \\
hlt    & 0   & 1   & 0    & 0  & 1    & 10  & 0    & 1000  & 1  & 0    & 256  & 0    & 1000 \\
pure c & 0   & 0   & 0    & 0  & 0    & 100 & 1000 & 0     & 1  & 0    & 256  & 4    & 1000 \\
pure r & 0   & 0   & 0    & 0  & 0    & 100 & 0    & 1000  & 1  & 0    & 256  & 0    & 1000 \\
sims:1 & 0   & 1   & 0    & 0  & 0    & 10  & 0    & 1000  & 1  & 0    & 256  & 0    & 1000 \\
sims:3 & 0   & 1   & 0    & 0  & 0    & 10  & 0    & -1000 & 1  & 0    & 256  & 4    & 1000 \\
sims:5 & 0   & 1   & 1    & 0  & 0    & 10  & 0    & 1000  & 1  & 0    & 256  & 0    & 1000 \\
sims:7 & 0   & 1   & 1    & 0  & 0    & 10  & 0    & -1000 & 1  & 0    & 256  & 4    & 1000 \\
sims:9 & 0   & 0   & 0    & 0  & 0    & 10  & 1000 & 0     & 1  & 0    & 256  & 4    & 1000 \\
\hline\hline
\end{tabular*}
\end{table}

Note that, apart from the \ttt{felsch:0} \amp \ttt{sims:9} strategies, all of
  the strategies are distinct, although some are very similar.
Further details of each strategy are contained in their entry in
  Section~\ref{sec:cmd}.

\section{Commands}\label{sec:cmd}

The following subsections give details of all the commands available when
  using the interactive interface.
The section headings match the help screen produced by the \ttt{help}
  command, and are in the same order.
%
Alternative forms of a command are separated by a \ttt{/}\kern-1pt, while
  any optional part of a command is denoted by \ttt{[\dots]}.
Case is not significant in command names, but that part of a command
  actually present must be correct, modulo white space.
The allowed parameter values are listed after a colon (\ttt{:}) either
  explicitly (e.g.~\ttt{1..7} means an integer in the range 1 to 7
  inclusive) or is one of the following:

  \bv
    \begin{tabular}{@{\ \ }l@{\quad}p{12cm}}
       \ttt{<int>}     &an integer;\\
       \ttt{<int list>}&a comma-separated list of \ttt{<int>};\\
       \ttt{<string>}  &an alphanumeric string (blanks allowed, but no
                        semicolons);\\
       \ttt{<filename>}&a \ttt{<string>} (but must be a valid UNIX filename);\\
       \ttt{<letter list>}&a list of lower-case letters, optionally separated
                        by blanks or commas;\\
       \ttt{<word list>}&a comma-separated list of \ttt{<word>}s;\\
       \ttt{<relator list>}&a comma-separated list of \ttt{<word>}s or
                        \ttt{<relation>}s;
    \end{tabular}
  \ev

  where a \ttt{<relation>} is an equals (\ttt{=}) separated list of
  \ttt{<word>}s, a (somewhat sketchy) BNF for \ttt{<word>} is given by

  \bv
     \begin{verbatim}
  <word>    = <factor> { "*" | "/" <factor> }
  <factor>  = <element> [ ["^"] <integer> | "^" <element> ]
  <element> = <generator> ["'"]
              | "(" <word> { "," <word> } ")" ["'"]
              | "[" <word> { "," <word> } "]" ["'"]
     \end{verbatim}
  \ev

  and a \ttt{<generator>} is a letter or an integer (see~\ref{sec:groupgen}).
A verbal description of a \ttt{<word list>} is given in~\ref{sec:grouprel}.
%
Appendix~\ref{app:ex} contains many examples of how to correctly drive
  \ace.

Parameters to a command are supplied after a colon (\ttt{:}).  
Each command is terminated by a newline or a semicolon (\ttt{;}), except
  in some cases where the argument may be a list of words, in which case
  newlines are ignored and a semicolon is the only terminator.
(E.g., the \ttt{add gen}, \ttt{add rel}, \ttt{rel} \amp \ttt{gen}
  commands.)
In many cases the parameters are optional, and entering the command
  without an argument prints the parameter's current value.
If the no-argument form has a special meaning, this is noted in its entry
  below.
Where there is no danger of confusion, the \ttt{:} and/or the \ttt{;} can
  usually be dispensed with.

\subsection{\ttt{add\ gen[erators]\ /\ sg\ :\ <word\ list>\ ;}}

Adds the words in the list to any subgroup generators already present.
The enumeration must be (re)started or redone, it cannot be continued.

\subsection{\ttt{add\ rel[ators]\ /\ rl\ :\ <relator\ list>\ ;}}

Adds the relators (words) or relations in the list to any relators 
already present.
The enumeration must be (re)started or redone, it cannot be continued.

\subsection{\ttt{aep\ :\ 1..7\ ;}}

The \ttt{aep} (all equivalent presentations) option runs an enumeration
  for each possible combination of relator ordering, relator rotations,
  and relator inversions.
As discussed by Cannon, Dimino, Havas \amp Watson \cite{CDHW} and Havas
  \amp Ramsay \cite{HR1} such equivalent presentations can yield large
  variations in the number of cosets required in an enumeration.
For this command, we are interested in this variation.

The \ttt{aep} option first performs a `priming run' using the parameters
  as they stand.
In particular, the \ttt{asis} \amp \ttt{mess} parameters are honoured.
It then turns \ttt{asis} on and \ttt{mess} off, and generates and tests
  the requested equivalent presentations.
The maximum and minimum values attained by \ttt{maxcos} \amp \ttt{totcos}
  are tracked, and each time a new `record' is found the relators used and
  the summary result line is printed.
At the end, some additional status information is printed: 
  the number of runs which yielded a finite index; 
  the total number of runs (excluding the priming run); 
  and the range of values observed for \ttt{maxcos} \amp \ttt{totcos}.
\ttt{asis} \amp \ttt{mess} are now restored to their original values, and 
  the system is ready for further commands.

The mandatory argument is considered as a binary number.
Its three bits are treated as flags, and control relator rotations (the
  $2^0$ bit), relator inversions (the $2^1$ bit) and relator orderings
  (the $2^2$ bit); ``1'' means `active' and ``0'' means
  `inactive'\kern-1.5pt.
The order in which the equivalent presentations are generated and tested
  has no particular significance, but note that the presentation as given
  (\kern-1.5pt\tit{after} the initial priming run) is the \tit{last}
  presentation to be generated and tested, so that the group's relators are
  left `unchanged' by running the \ttt{aep} option.

As an example (drawn from the discussion in \cite{HR1}) consider the index
  $448$ enumeration of the right cosets of $\langle a^2,Ab \rangle$ in
  $(8,7 \mid 2,3)$, where 
  $$ (8,7 \mid 2,3) 
    = \langle a,b \mid a^8 = b^7 = (ab)^2 = (Ab)^3 = 1 \rangle . $$
There are $4!=24$ relator orderings and $2^4=16$ combinations of relator or
  inverted relator.
Exponents are taken into account when rotating relators, so the relators
  given give rise to 1, 1, 2 \amp 2 rotations respectively, for a total
  of $1.1.2.2=4$ combinations.
So the command \ttt{aep:7} would generate and test $24.16.4=1536$ 
  equivalent presentations, while \ttt{aep:3} would generate and test 
  $16.4=64$ equivalent presentations.

%\tsc{Guru Note:}
\tsc{Notes:}
There is no way to stop the \ttt{aep} option before it has completed,
  other than killing the task.
So do a reality check beforehand on the size of the search space and the 
  time for each enumeration.
If you are interested in finding a `good' enumeration, it can be very
  helpful, in terms of running time, to put a tight limit on the number of
  cosets via the \ttt{max} parameter.
(You may also have to set \ttt{com:100} to prevent time-wasting attempts
  to recover space via compaction.)
This maximises throughput by causing the `bad' enumerations, which are in
  the majority, to overflow quickly and abort.
If you wish to explore a very large search-space, consider firing up many
  copies of \ace, and starting each with a `random' equivalent
  presentation.
Alternatively, you could use the \ttt{rep} command.

\subsection{\ttt{ai\ /\ alter\ i[nput]\ :\ [<filename>]\ ;}}

By default, commands to \ace\ are read from the standard input file
  (i.e., the `keyboard'\kern-1.5pt, \ttt{stdin}).
The \ttt{ai} command closes the current input file, and opens
  \ttt{<filename>} as the source of commands.
If \ttt{<filename>} can't be opened, input reverts to \ttt{stdin}.

\tsc{Notes:}
If you switch to taking input from (another) file, remember to switch back
  before the end of that file; otherwise the \ttt{EOF} there will cause
  \ace\ to terminate.

\subsection{\ttt{ao\ /\ alter\ o[utput]\ :\ [<filename>]\ ;}}

By default, output from \ace\ is sent to the standard output file
  (i.e., the `terminal'\kern-1.5pt, \ttt{stdout}).
The \ttt{ao} command closes the current output file, and opens
  \ttt{<filename>} for all future output.
If \ttt{<filename>} can't be opened, output reverts to \ttt{stdout}.

\subsection{\ttt{as[is]\ :\ [0/1]\ ;}}

By default, \ace\ freely \amp cyclically reduces the relators, freely
  reduces the subgroup generators, and sorts relators \amp generators
  in length-increasing order (a stable insertion sort is used).
If you do not want this, you can switch it off by \ttt{asis:1}.

\tsc{Notes:}
As well as allowing you to process the presentation in the form given, this
  is useful for forcing definitions to be made in a prespecified order, by
  introducing dummy (i.e., freely trivial) subgroup generators.
Note also that the exact form of the presentation can have a significant
  impact on the enumeration statistics; it is not the case that the 
  default option always yields the best enumeration.

\tsc{Guru Notes:}
When \ttt{asis:0}, a (reduced) relator of the form \ttt{aa} or \ttt{AA} 
  causes that generator to be treated as an involution.
In the relators and subgroup generators, the inverses of involutionary
  generators are automatically replaced with the generator.
When \ttt{asis:1}, only relators of the form \ttt{a\^{}2} cause the 
  generator to be treated as an involution.
The forms \ttt{aa} \amp \ttt{a\^{}2} are preserved in any printout, so
  that you can track \ace's behaviour.

\subsection{\ttt{beg[in]\ /\ end\ /\ start\ ;}}

Start an enumeration.
Any existing information in the table is cleared, and the enumeration
  starts from coset \#1 (i.e., the subgroup).

\subsection{\ttt{bye\ /\ exit\ /\ q[uit]\ ;}}

The exits \ace\ nicely, printing the date and the time.
An \ttt{EOF} (end-of-file; i.e., \ttt{\^{}d}) has the same effect, so 
  proper termination occurs if \ace\ is taking its input from a script
  file.

\subsection{\ttt{cc\ /\ coset\ coinc[idence]\ :\ <int>\ ;}}

Print out the representative of coset \#\ttt{<int>}, and add it to the
  subgroup generators; i.e., equates this coset with coset \#1, the
  subgroup.

\subsection{\ttt{c[factor]\ /\ ct[\ factor]\ :\ [<int>]\ ;}}

The value of this parameter sets the `blocking factor' for C-style
  definitions; i.e., the number of coset definitions made by filling the
  next empty coset table position during each pass through the enumerator's
  main loop.
The absolute value of \ttt{<int>} is the value used.
The enumeration style is selected by the values of the \ttt{ct} \amp
  \ttt{rt} parameters; see Section~\ref{sec:style}.

\subsection{\ttt{check\ /\ redo\ ;}}

An extant enumeration is redone, using the current parameters.
Any existing information in the table is retained, and the enumeration
  is restarted from coset \#1 (i.e., the subgroup).

\tsc{Notes:}
This option is really intended for the case where additional relators
  and/or subgroup generators have been introduced.
The current table, which may be incomplete or exhibit a finite index, is
  still `valid'\kern-1.5pt.
However, the additional data may allow the enumeration to complete, or
  cause a collapse to a smaller index.

\subsection{\ttt{com[paction]\ :\ [0..100]\ ;}}

The key word \ttt{com} controls compaction of the coset table during an
  enumeration.
Compaction recovers the space allocated to cosets which are flagged as 
  dead (i.e., which were found to be coincident with lower-numbered
  cosets). 
It results in a table where all the active cosets are numbered contiguously
  from \#1, and with the remainder of the table available for new cosets.

The coset table is compacted when a coset definition is required, there is
  no space for a new coset available, and provided that the given 
  percentage of the coset table contains dead cosets.
For example, \ttt{com:20} means compaction will occur only if 20\% or more
  of the cosets in the table are dead.
The argument can be any integer in the range 0--100, and the default value
  is 10 or 100; see Section~\ref{sec:strat}.
An argument of 100 means that compaction is never performed, while an
  argument of 0 means always compact, no matter how few dead cosets there
  are (provided there is at least one, of course).

Compaction may be performed multiple times during an enumeration, and the
  table that results from an enumeration may or may not be compact,
  depending on whether or not there have been any coincidences since the
  last compaction (or from the start of the enumeration, if there have been
  no compactions).
If messaging is enabled (i.e., $\ttt{mess} \ne 0$), then a progress
  message (labelled \ttt{CO}) is printed each time the compaction routine
  is actually called (as opposed to each time it is potentially called).

\tsc{Notes:}
In some strategies (e.g., HLT) a lookahead phase may be run before
  compaction is attempted.
In other strategies (e.g., \ttt{sims:3}) compaction may be performed while
  there are outstanding deductions; since deductions are discarded during
  compaction, a final \ttt{RA} phase will (automatically) be performed.
%
Compacting a table `destroys' information and history, in the sense that
  the table entries for any dead cosets are deleted, along with their
  coincidence list data.
At Level 2, it is not possible to access the `data' in dead cosets; in
  fact, most options that require table data compact the table 
  automatically before they run.

\subsection{\ttt{cont[inue]\ ;}}

Continue the current enumeration, building upon the existing table.
If a previous run stopped without producing a finite index you can, in
  principle, change any of the parameters and continue on.
Of course, if you make any changes which invalidate the current table, you
  won't be allowed to continue, although you may be allowed to redo.

\subsection{\ttt{cy[cles]\ ;}}

Print out the table in cycles; i.e., the permutation representation.

\subsection{\ttt{ded\ mo[de]\ /\ dmod[e]\ :\ [0..4]\ ;}}

A completed table is only valid if every table entry has been tested in
  all essentially different positions in all relators.
This testing can either be done directly (Felsch strategy) or via relator
  scanning (HLT strategy).
If it is done directly, then more than one deduction (i.e., table entry)
  can be outstanding at any one time.
So the untested deductions are stored in a stack.
Normally this stack is fairly small but, during a collapse, it can become
  very large.

This command allows the user to specify how deductions should be handled.
The options are:
\ttt{0}, discard deductions if there is no stack space left;
\ttt{1}, as \ttt{0}, but purge redundant cosets off the top of the stack
  at every coincidence;
\ttt{2}, as \ttt{0}, but purge all redundant cosets from the stack at 
  every coincidence;
\ttt{3}, discard the entire stack if it overflows;
\ttt{4}, if the stack overflows, then double the stack size and purge all
  redundant cosets from the stack.

The default deduction mode is either $0$ or $4$, depending on the strategy
  selected (see Section~\ref{sec:strat}), and it is recommended that you
  stay with the default.
If you want to know more details, read the code.

\tsc{Notes:}
If deductions are discarded for any reason, then a final \ttt{RA} phase
  will be run automatically at the end of the enumeration, if necessary, to
  check the result.

\subsection{\ttt{ded\ si[ze]\ /\ dsiz[e]\ :\ [0/1..]\ ;}}

Sets the size of the (initial) allocation for the deduction stack.
The size is in terms of the number of deductions, with one deduction
  taking two words (i.e., 8 bytes).
The default size, of $1000$, can be selected by a \ttt{0} argument.
See the \ttt{dmod} entry for a (brief) discussion of deduction handling.

\subsection{\ttt{def[ault]\ ;}}

This selects the default strategy, which is based on the defaulted
  R/C style; see Sections~\ref{sec:style} \amp \ref{sec:strat}.
The idea here is that we assume that the enumeration is `easy'\kern-1.5pt,
  and start out in R style.
If it turns out not to be easy, then we regard it as `hard'\kern-1.5pt,
  and switch to CR style, after performing a lookahead on the entire
  table.

\subsection{\ttt{del\ gen[erators]\ /\ ds\ :\ <int>\ list>\ ;}}

This command allows subgroup generators to be deleted from the
  presentation.
If the generators are numbered from one in the output of, say, the
  \ttt{sr} command, then the generators listed in \ttt{<int\ list>} are
  deleted; \ttt{<int\ list>} must be a strictly increasing sequence.

\subsection{\ttt{del\ rel[ators]\ /\ dr\ :\ <int>\ list>\ ;}}

As \ttt{del gen}, but for the group's relators.

\subsection{\ttt{d[ump]\ :\ [0/1/2[,0/1]]\ ;}}

Dumps the internal variables of \ace.
The first argument selects which of the three levels of \ace\ to dump, and
  defaults to Level $0$.
The second argument selects the level of detail, and defaults to $0$
  (i.e., less detail).
This option is intended for gurus; the source code should be consulted to
  see what the output means.

\subsection{\ttt{easy\ ;}}

If this strategy is selected, we run in R style (i.e., HLT) and turn
  lookahead \amp compaction off.
For small and/or easy enumerations, this mode is likely to be the fastest.

\subsection{\ttt{echo\ :\ [0/1]\ ;}}

By default, \ace\ does not echo its commands.
If you wish it to do so, turn this feature on with \ttt{echo:1}.
This feature can be used to render output files from \ace\ less
  incomprehensible.

\subsection{\ttt{enum[eration]\ /\ group\ name\ :\ <string>\ ;}}

This command defines the name by which the current enumeration (i.e.,
  the group being used) will be identified in any printout.
It has no effect on the actual enumeration, and defaults to \ttt{G}.
An empty name is accepted; to see what the current name is, use the
  \ttt{sr} command.

\subsection{\ttt{fel[sch]\ :\ [0/1]\ ;}}

An argument of \ttt{0} or no argument selects the Felsch strategy, while
  an argument of \ttt{1} selects Felsch with all relators in the subgroup
  and turns gap-filling on; see Section~\ref{sec:strat}.

\subsection{\ttt{f[factor]\ /\ fi[ll\ factor]\ :\ [0/1..]\ ;}}

If gap-filling is on, then gaps of length one found during deduction
  testing are preferentially filled (see \cite{Hav}).
However, this potentially violates the formal requirement that all rows
  in the table are eventually filled (and tested against the relators).
The fill factor is used to ensure that some constant proportion of the
  coset table is always kept filled.
Before defining a coset to fill a gap of length one, the enumerator checks 
  whether \ttt{fi} times the completed part of the table is at least the
  total size of the table and, if not, fills rows in standard order
  instead of the gap.

An argument of \ttt{0} selects the default value of
  $\lfloor 5(n+2)/4 \rfloor$, where $n$ is the number of columns in the
  table.
All other things being equal, we'd expect the ratio of the total size
  of the table to the completed part of the table to be $n+1$, so the 
  default fill factor allows a moderate amount of gap-filling.

\tsc{Notes:}
If \ttt{fi} is smaller than $n$, then there is generally no gap-filling,
  although its processing overhead is still incurred.
A large value of \ttt{fi} can cause infinite looping.
However, in general, a large value does work well.
The effects of the various gap-filling stategies vary widely.  
It is not clear which values are good general defaults or, indeed, whether
  any strategy is always `not too bad'\kern-1.5pt.

\subsection{\ttt{gen[erators]\ /\ subgroup\ gen[erators]\ 
  :\ <word\ list>\ ;}}

By default, there are no subgroup generators and the subgroup is trivial.
This command allows a list of subgroup generating words to be entered.

\subsection{\ttt{gr[oup\ generators]:\ [<letter\ list>\ /\ <int>]\ ;}}
  \label{sec:groupgen}

This command introduces the group generators, which may be represented in
  one of two ways.
They may be presented as a list of lower-case letters, optionally
  separated by commas.
This is the usual method, and gives up to $26$ generators.
Subsequently, upper-case letters can be used, if desired, to stand for the
  inverse of their lower-case versions; 
  e.g., \ttt{A} for \ttt{a\^{}-1}, \ttt{B} for \ttt{b\^{}-1}, etc.
Alternatively, a positive integer can be used to indicate the number of
  generators.
For example, \ttt{gr:5} indicates that there are five generators, 
  designated \ttt{1}, \ttt{2}, \ttt{3}, \ttt{4} \amp \ttt{5}.

\tsc{Notes:}
Any use of the \ttt{gr} command which actually defines generators
  invalidates any previous enumeration, and stays in effect until the next
  \ttt{gr} command.
Any words for the group or subgroup must be entered using the nominated
  generator format, and all printout will use this format.
%
This command is not optional, nor is their any default.
A valid set of generators is the minimum information necessary before
  \ace\ will attempt an enumeration.

\tsc{Guru Notes:}
The columns of the coset table are allocated in the same order as the 
  generators are listed, insofar as this is possible, given that the first
  two columns must be a generator/inverse pair or a pair of involutions.
The ordering of the columns can, in some cases, effect the definition
  sequence of cosets and impact the statistics of an enumeration.

\subsection{\ttt{group\ relators\ /\ rel[ators]\ :\ <relator\ list>\ ;}}
  \label{sec:grouprel}

By default, or if an empty argument to this command is used, the group
  is free.
Otherwise, this command is used to introduce the group's defining
  relators.
In order to allow \ace\ to accept presentations from a variety of
  sources, many kinds of word representations are allowed. 
\ace\ accepts words in the nominated generators, allowing \ttt{*} for
  multiplication, \ttt{\^{}} for exponentiation and conjugation, and
  brackets for precedence specification. 
Round or square brackets may be used for commutation. 
(There is no danger of confusion between \ttt{[a,b]}/\ttt{(a,b)} and
  \ttt{(ab)}, since a \ttt{,} implies commutation, while no comma implies a
  word.)
If letter generators are used, multiplication and exponentiation signs
  (but \tit{not} conjugation signs) may be omitted; e.g., \ttt{a3} is the
  same as \ttt{a\^{}3} and \ttt{ab} is the same as \ttt{a*b}.
Also, the exponent \ttt{-1} can be abbreviated to \ttt{-},
  so \ttt{a-} stands for \ttt{A}.
Inverses can also be denoted by \ttt{'} or \ttt{/}, so
  $w_1w_2\ttt{'} = w_1\ttt{/}w_2 = w_1w^{-1}_2$\kern-1.5pt.
The \ttt{*} can also be dropped for numeric generators; but of course two
  numeric generators, or a numeric exponent and a numeric generator, must
  be separated by whitespace. 
Remember that \ttt{A} stands for \ttt{a\^{}-1}, \ttt{a\^{}b} for
  \ttt{Bab} and \ttt{[a,b]} \amp \ttt{[a,b,c]} for \ttt{ABab} \amp
  \ttt{[[a,b],c]}.

\ttt{<relator\ list>} is a comma-separated list of words (relators)
  or relations.
A relation is a list of equated words, e.g.~$w_1=w_2=w_3$ (equivalent to
  the relators $w_1w^{-1}_2$ and $w_1w^{-1}_3$).

\subsection{\ttt{hard\ ;}}

In many `hard' enumerations, a mixture of R-style and C-style definitions,
  all tested in all essentially different positions, is appropriate.
This option selects such a mixed strategy; see Section~\ref{sec:strat}.
The idea here is that most of the work is done C-style (with the relators
  in the subgroup and with gap-filling active), but that every $1000$
  C-style definitions a further coset is applied to all relators.

\tsc{Guru Notes:}
The \ttt{1000}/\ttt{1} mix is not necessarily optimal, and some
  experimentation may be needed to find an acceptable balance (see, for
  example, \cite{HR1}).
Note also that, the longer the total length of the presentation, the more
  work needs to be done for each coset application to the relators; one
  coset application can result in more than $1000$ definitions, reversing
  the balance between R-style and C-style definitions.

\subsection{\ttt{h[elp]\ ;}}

Prints the help screen; i.e., all the subsection headings in this section.
Note that this list is fairly long, so its top may disappear off the top 
  of the screen.

\subsection{\ttt{hlt\ ;}}

Selects the standard HLT strategy; see Section~\ref{sec:strat}.

\subsection{\ttt{ho[le\ limit]\ :\ [-1/0..100]\ ;}}

An experimental feature which allows an enumeration to be terminated when
  the percentage of holes in the table exceeds a given value.
In practice, calculating this is very expensive, and it tends to remain
  constant or decrease throughout an enumeration.
So the feature doesn't seem very useful.
The default value of \ttt{-1} turns this feature off.
If you want more details, read the source code.

\subsection{\ttt{look[ahead]\ :\ [0/1..4]\ ;}}

Although HLT-style strategies are fast, they are local, in the sense that
  the implications of any definitions/deductions made while applying cosets
  may not become apparent until much later.
One way to alleviate this problem is to perform lookaheads occasionally;
  that is, to test the information in the table, looking for deductions
  or concidences.
\ace\ can perform a lookahead when the table overflows, before the
  compaction routine is called.
An argument of \ttt{0} disables lookahead.
Lookahead can be done using the entire table or only that part of the table
  above the current coset, and it can be done R-style (scanning cosets
  from the beginning of relators) or C-style (testing all definitions in
  all essentially different positions).
An argument of \ttt{1} does a partial table, R-style lookahead; \ttt{2}
  does all the table, C-style; \ttt{3} does all the table, R-style; and 
  \ttt{4} does a partial table, C-style.
The default is either \ttt{0} or \ttt{1}; see Section~\ref{sec:strat}.

\tsc{Notes:}
A lookahead can do a significant amount of work, so this phase may take a
  long time.
The value of \ttt{mend} is honoured during R-style lookaheads.

\subsection{\ttt{loop[\ limit]\ :\ [0/1..]\ ;}}

The core enumerator is organised as a state machine, with each step
  performing an `action' (i.e., lookahead, compaction) or a block of
  actions (i.e., $|\ttt{ct}|$ coset definitions, $|\ttt{rt}|$ coset
  applications).
The number of passes through the main loop (i.e., steps) is counted, and
  the enumerator can make an early return when this count hits the
  value of \ttt{loop}.
A value of \ttt{0}, the default, turns this feature off.

\tsc{Guru Notes:}
You can do lots of really neat things using this feature, but you need
  some understanding of the internals of \ace\ to get real benefit from
  it.
Read the code!

\subsection{\ttt{max[\ cosets]\ :\ [0/2..]\ ;}}

By default, all of the workspace is used, if necessary, in building the
  coset table.
So the table size is an upper bound on how many cosets can be active at
  any one time.
The \ttt{max} option allows a limit to be placed on how much of the
  physical table space is made available to the enumerator.
Enough space for at least two cosets (i.e., the subgroup and one other)
  must be made available.
An argument of \ttt{0} selects all of the workspace.

\subsection{\ttt{mend[elsohn]\ :\ [0/1]\ ;}}

Mendelsohn style processing during relator scanning/closing is turned on by
  \ttt{mend:1} and off by \ttt{mend:0}.
Off is the default, and here coset applications are done only at the start
  (and end) of a relator.
Mendelsohn on means that coset applications are done at all cyclic
  permutations of the (base) relator.
The effect of the Mendelsohn parameter is case-specific.
It can mean the difference between success or failure, or it can impact
  the number of cosets required, or it can have no effect on an
  enumeration's statistics.

\tsc{Notes:}
Processing all cyclic permutations of the relators can be very
  time-consuming, especially if the presentation if large.
So, all other things being equal, the Mendelsohm flag should normally be
  left off.
Note that Mendelsohn's paper \cite{Men} discusses tracing all cyclic 
  shifts of both the relators and their formal inverses.
\ace\ only processes the relators.
However, since relators are scanned from both the front and the rear, we
  effectively process the inverses.
%TBA: \\
% cyclic stuff in \ttt{index}, \tbf{not} equivalent to EDF!

\subsection{\ttt{mess[ages]\ /\ mon[itor]\ :\ [<int>]\ ;}}

By default, or if the argument is \ttt{0}, \ace\ prints out only a single
  line of information giving the result of each enumeration.
It \ttt{mess} is non-zero then the presentation \amp the parameters are
  echoed at the start of the run, and messages on the enumeration's status
  as it progresses are also printed out.
The absolute value of \ttt{<int>} sets the frequency of the progress
  messages, with a negative sign turning hole monitoring on.
The initial printout of the presentation \amp the parameters is the same
  as that produced by the \ttt{sr:1} command; see Appendix~\ref{app:ex}
  for some examples.

The result line gives the result of the call to the enumerator and some
  basic statistics (see Appendix~\ref{app:ex} for some examples).
The possible results are given in Table~\ref{tab:rslts}; any result not
  listed represents an internal error and should be reported.
The statistics given are, in order: 
  \ttt{a}, number of active cosets; 
  \ttt{r}, number of applied cosets;
  \ttt{h}, first (potentially) incomplete row;
  \ttt{n}, next coset definition number; 
  \ttt{h} (if $\ttt{mess} < 0$), percentage of holes in the table;
  \ttt{l}, number of main loop passes;
  \ttt{c}, total CPU time;
  \ttt{m}, maximum active cosets;
  and \ttt{t}, total cosets defined.

\begin{table}
\hrule
\caption{Possible enumeration results}
\label{tab:rslts}
\smallskip
\renewcommand{\arraystretch}{0.875}
\begin{tabular*}{\textwidth}{@{\extracolsep{\fill}}lll} 
\hline\hline
result & level & meaning \\
\hline
\ttt{INDEX = x}         & 0 & finite index of \ttt{x} obtained \\
\ttt{OVERFLOW}          & 0 & out of table space \\
\ttt{SG PHASE OVERFLOW} & 0 & out of space (processing subgroup
                                generators) \\
\ttt{ITERATION LIMIT}   & 0 & \ttt{loop} limit triggered \\
\ttt{TIME LIMT}         & 0 & \ttt{ti} limit triggered \\
\ttt{HOLE LIMIT}        & 0 & \ttt{ho} limit triggered \\
\ttt{INCOMPLETE TABLE}  & 0 & all cosets applied, but table has holes \\
\ttt{MEMORY PROBLEM}    & 1 & out of memory (building data structures) \\
\hline\hline
\end{tabular*}
\end{table}

The progress message lines consist of an initial tag, some fixed
  statistics, and some variable statistics.
The possible message tags are listed in Table~\ref{tab:prog}, along
  with their meanings.
The tags indicate the function just completed by the enumerator.
The tags with a `y' in the `action' column represent functions which are
  aggregated and counted.
Every time this count overflows the value of \ttt{mess}, a message line
  is printed and the count is zeroed.
Those tags flagged with a `y*' are only present if the appropriate option
  has been included in the build (see the \ttt{opt} command).
Tags with an `n' in the `action' column are not counted, and cause a
  message line to be output every time they occur.
They also cause the action count to be reset.

The fixed portion of the statistics consists of the \ttt{a}, \ttt{r}, 
  \ttt{h}, \ttt{n}, \ttt{h}, \ttt{l} \amp \ttt{c} values, as for the
  result line, except that \ttt{c} is the time since the previous message
  line. 
The variable portion of the statistics can be:
  the \ttt{m} \amp \ttt{t} values, as for the result line;
  \ttt{d}, the current size of the deduction stack;
  \ttt{s}, \ttt{d} \amp \ttt{c} (with \ttt{DS} tag), the new stack size,
    the non-redundant deductions retained, and the redundant deductions
    discarded.

\begin{table}
\hrule
\caption{Possible progress messages}
\label{tab:prog}
\smallskip
\renewcommand{\arraystretch}{0.875}
\begin{tabular*}{\textwidth}{@{\extracolsep{\fill}}lll} 
\hline\hline
message & action & meaning \\
\hline
\ttt{AD} & y  & coset \#1 application definition 
                        (\ttt{SG}/\ttt{RS} phase) \\
\ttt{RD} & y  & R-style definition \\
\ttt{RF} & y  & row-filling definition \\
\ttt{CG} & y  & immediate gap-filling definition \\
\ttt{CC} & y* & coincidence processed \\
\ttt{DD} & y* & deduction processed \\
\ttt{CP} & y  & preferred list gap-filling definition \\
\ttt{CD} & y  & C-style definition \\
\ttt{Lx} & n  & lookahead performed (type \ttt{x}) \\
\ttt{CO} & n  & table compacted \\
\ttt{CL} & n  & complete lookahead (table as deduction stack) \\
\ttt{UH} & n  & updated completed-row counter \\
\ttt{RA} & n  & remaining cosets applied to relators \\
\ttt{SG} & n  & subgroup generator phase \\
\ttt{RS} & n  & relators in subgroup phase \\
\ttt{DS} & n  & stack overflowed (compacted and doubled) \\
\hline\hline
\end{tabular*}
\end{table}

\tsc{Notes:}
Hole monitoring is expensive, so don't turn it on unless you really need
  it.
If you wish to print out the presentation \amp the parameters, but not
  the progress messages, then set \ttt{mess} non-zero, but very large.
(You'll still get the \ttt{SG}, \ttt{DS}, etc.\@ messages, but not the
  \ttt{RD}, \ttt{DD}, etc.\@ ones.)
You can set \ttt{mess} to \ttt{1}, to monitor all enumerator actions, but
  be warned that this can yield very large output files.

\subsection{\ttt{mo[de]\ ;}}

Prints the possible enumeration modes, as deduced from the command history
  since the last call to the enumerator; see Section~\ref{sec:style}.

\subsection{\ttt{nc\ /\ normal[\ closure]\ :\ [0/1]\ ;}}

If the argument is missing or \ttt{0}, test for normal closure.
If a subgroup generator does not normalise a generator, then this is
  printed out.
If the argument is \ttt{1}, then any non-normalising conjugates are
  also added to the subgroup generators.

\subsection{\ttt{no[\ relators\ in\ subgroup]\ :\ [-1/0/1..]\ ;}}

It is sometimes helpful to include the relators in the subgroup, in the
  sense that they are applied to coset \#1 at the start of an enumeration.
An argument of \ttt{0} turns this feature off, and an argument of \ttt{-1}
  includes all the relators.
A positive argument includes the appropriate number of relators, in order.

\subsection{\ttt{opt[ions]\ ;}}

This command dumps details of the options included in the version of \ace\
  you're running; i.e., what compiler flags were set when the executable
  was built.
A typical output, illustrating the default build, is:

\bv\begin{verbatim}
Executable built:
  Sat Feb 27 15:57:59 EST 1999
Level 0 options:
  statistics package = on
  coinc processing messages = on
  dedn processing messages = on
Level 1 options:
  workspace multipliers = decimal
Level 2 options:
  host info = on
\end{verbatim}\ev

\subsection{\ttt{oo\ /\ order[\ option]\ :\ <int>\ ;}}

This option finds a coset with order a multiple of $|\ttt{<int>}|$ modulo
  the subgroup, and prints out its coset representative.
If $\ttt{<int>} < 0$, then all cosets meeting the requirement are printed.
If $\ttt{<int>} = 0$, then the orders of all cosets are printed.
If $\ttt{<int>} > 0$, then the first coset meeting the requirement is
  printed.

\subsection{\ttt{par[ameters]\ ;}}

An old option, which did nothing.
It is included for backward compatability.
Pre-\acet\ scripts may contain this option, which is quietly ignored
  by \acet.

\subsection{\ttt{path[\ compression]\ :\ [0/1]\ ;}}

To correctly process multiple concidences, a union-find must be performed.
If both path compression and weighted union are used, then this can be
  done in essentially linear time (see, e.g., \cite{CLR}).
Weighted union alone, in the worst-case, is worse than linear, but is
  subquadratic.
In practice, path compression is expensive, since it involves many coset
  table accesses.
So, by default, path compression is turned off; it can be turned on by
  \ttt{path:1}.
It has no effect on the result, but may effect the running time and the
  internal statistics.

\tsc{Guru Notes:}
The whole question of the best way to handle large coincidence forests is
  problematic.
Formally, \ace\ does not do a weighted union, since it is constrained to
  replace the higher-numbered of a coincident pair.
In practice, this seems to amount to much the same thing!
Turning path compression on cuts down the amount of data movement during
  coincidence processing at the expense of having to trace the paths and
  compress them.
In general, it does not seem to be worthwhile.

\subsection{\ttt{pd\ mo[de]\ /\ pmod[e]\ :\ [0/1..3]\ ;}}

If the argument is \ttt{0}, then Felsch style definitions are made using
  the next empty table slot.
If the argument is non-zero, then gaps of length one found during relator
  scans in Felsch style are preferentially filled (subject to the value
  of \ttt{fi}).
If the argument is \ttt{1}, they are filled immediately, and if it is
  \ttt{2}, the consequent deduction is also made immediately (of course,
  these are also put on the deduction stack).  
If the argument is \ttt{3}, then the gaps are noted in the preferred
  definition queue.  
Provided a live such gap survives (and no coincidence occurs, which causes
  the queue to be discarded) the next coset will be defined to fill the
  oldest gap of length one.
The default value is either \ttt{0} or \ttt{3}, depending on the strategy
  selected (see Section~\ref{sec:strat}).
If you want to know more details, read the code.

\subsection{\ttt{pd\ si[ze]\ /\ psiz[e]\ :\ [0/2/4/8/...]\ ;}}

The preferred definition queue is implemented as a ring, dropping
  earliest entries.
Its size \tit{must} be $2^n$\kern-2pt, for some $n>0$.
An argument of \ttt{0} selects the default size of $256$.
Each queue slot takes two words (i.e., 8 bytes), and the queue can store
  up to $2^n-1$ entries.

\subsection{\ttt{print\ det[ails]\ /\ sr\ :\ [0/1]\ ;}}

This command prints out details of the current presentation and parameters.
No argument, or an argument of \ttt{0}, prints out the group \amp
  subgroup name, the group's relators and the subgroup's generators.
If the argument is \ttt{1}, then the current setting of the parameters is
  also printed.
The printout is the same as that produced at the start of a run when
  messaging is on.
See Appendix~\ref{app:ex} for some examples.

\tsc{Notes:}
The output is printed out in a form suitable for input, so that a record
  of a previous run can be used to replicate the run.
Note that, due to the defaulting of some parameters and the special
  meaning attached to some values, a little care has to be taken in
  interpreting the parameters.
If you wish to \tit{exactly} duplicate a run, you should use the output
  of \ttt{sr} \tit{after} the run completes.

\subsection{\ttt{pr[int\ table]\ :\ [[-]<int>[,<int>[,<int>]]]\ ;}}

Compact the table, and then print it out from the first to the second
  argument, in steps of the third argument.
If the first argument is negative, then the orders (if available) and
  representatives of the cosets are printed also.
The third argument defaults to one.
The one-argument form is equivalent to the two-argument form with a first
  argument of \ttt{1} and the argument used as the second argument.
The no-argument form prints the entire table, without orders or
  representatives.

\subsection{\ttt{pure\ c[t]\ ;}}

Sets the strategy to basic C-style (coset table based) -- 
  no compaction, no gap-filling, no relators in subgroup; 
  see Section~\ref{sec:strat}.

\subsection{\ttt{pure\ r[t]\ ;}}

Sets the strategy to basic R-style (relator based) -- 
  no Mendelsohn, no compaction, no lookahead, no row-filling;
  see Section~\ref{sec:strat}.

\subsection{\ttt{rc\ /\ random\ coinc[idences]:\ <int>[,<int>]\ ;}}

This option attempts to find nontrivial subgroups with index a multiple
  of the first argument by repeatedly putting random cosets coincident
  with coset \#1 and seeing what happens.
The starting coset table must be non-empty, but need not be complete.
The second argument puts a limit on the number of attempts, with a default
  of eight.
For each attempt, we repeatedly add random coset representatives to the 
  subgroup and redo the enumeration.
If the table becomes too small, the attempt is aborted, the original 
  subgroup generators restored, and another attempt made.
If an attempt succeeds, then the new set of subgroup generators is
  retained.

\tsc{Guru Notes:}
A coset can have many different representatives.
Consider running \ttt{st} before \ttt{rc}, to canonicise the table and the
  representatives.

\subsection{\ttt{rec[over]\ /\ contig[uous]\ ;}}

Invokes the compaction routine on the table to recover the space used by
  any dead cosets.
A \ttt{CO} message line is printed if any cosets were recovered, and a
  \ttt{co} line if none were.
This routine is called automatically if the \ttt{cy}, \ttt{nc}, \ttt{pr}
  or \ttt{st} options are invoked.

\subsection{\ttt{rep\ :\ 1..7[,<int>]\ ;}}

The \ttt{rep} (random equivalent presentations) option complements the
  \ttt{aep} option.
It generates and tests some random equivalent presentations.
The mandatory argument acts as for \ttt{aep}, while the optional second
  argument sets the number of presentations, with a default of eight.

The routine first turns \ttt{asis} on and \ttt{mess} off, and then 
  generates and tests the requested equivalent presentations.
For each presentation the relators used and the summary result line is
  printed.
\ttt{asis} \amp \ttt{mess} are now restored to their original values, and 
  the system is ready for further commands.

\tsc{Notes:}
The relator inversions \amp rotations are `genuinely' random.
The relator permuting is a little bit of a kludge, with the `quality' of
  the permutations tending to improve with successive presentations. 
When the \ttt{rep} command completes, the presentation active is the 
  \tit{last} one generated.

\tsc{Guru Note:}
It might appear that neglecting to restore the original presentation is an
  error.
In fact, it is a useful feature!
Suppose that the space of equivalent presentations is too large to
  exhaustively test.
As noted in the entry for \ttt{aep}, we can start up multiple copies of
  \ttt{aep} at random points in the search-space.
Manually generating `random' equivalent presentations to serve as
  starting-points is tedious and error-prone.
The \ttt{rep} option provides a simple solution; simply run \ttt{rep}
  before \ttt{aep}!

\subsection{\ttt{restart\ ;}}

An old option, included for backward compatability.
Use the \ttt{check/redo} option instead.
Pre-\acet\ scripts may contain this option, which is quietly ignored
  by \acet.

\subsection{\ttt{r[factor]\ /\ rt[\ factor]\ :\ [<int>]\ ;}}

The value of this parameter sets the `blocking factor' for R-style
  definitions; i.e., the number of cosets applied to all the relators
  during each pass through the enumerator's main loop.
The absolute value of \ttt{<int>} is the value used.
The enumeration style is selected by the values of the \ttt{ct} \amp
  \ttt{rt} parameters; see Section~\ref{sec:style}.

\subsection{\ttt{row[\ filling]\ :\ [0/1]\ ;}}

When making HLT-style definitions, it is normal to scan each row of the
  table after its coset has been applied to all relators, and make 
  definitions to fill any holes encountered.
Failure to do so can cause even simple enumerations to overflow; see
  Section~\ref{ex002}.
To turn row filling off, use \ttt{row:0}.
  
\subsection{\ttt{sc\ /\ stabil[ising\ cosets]\ :\ <int>\ ;}}

This option prints out the stabilising cosets of the subgroup.
If $\ttt{<int>} > 0$, it prints the first \ttt{<int>} stabilising cosets.
If $\ttt{<int>} = 0$, it prints all of the stabilising cosets, plus their
  representatives.
If $\ttt{<int>} < 0$, it prints the first $|\ttt{<int>}|$ stabilising cosets,
  plus their representatives.

\subsection{\ttt{sims\ :\ 1/3/5/7/9\ ;}}

In his book \cite{Sim}, Sims discusses ten standard enumeration
  strategies.
These are effectively HLT (with \amp without the \ttt{mend} parameter,
  and with \amp without continuous `lookahead') and Felsch, all either with
  or without table standardisation as the enumeration proceeds.
\ace\ does not implement table standardisation on an ongoing basis, 
  although tables from an incomplete or paused enumeration can be 
  standardised before the enumeration is continued.
The other five strategies are implemented, and can be selected by this
  command.
The argument matches the number given in \cite[\S5.5]{Sim}; see
  Section~\ref{sec:strat} for the parameter settings.
With care, it is possible to duplicate the statistics given in \cite{Sim};
  some examples are given in Section~\ref{ex001}.

\subsection{\ttt{st[andard\ table]\ ;}}

This option compacts and then standardises the table.
That is, for a given ordering of the generators in the columns of the
  table, it produces a canonic table.
In such a table, a row-major scan encounters previously unseen cosets in
  numeric order; see Section~\ref{ex000} for an example.

\tsc{Notes:}
In a canonic table, the coset representatives are in length plus (column
  order) lexicographic order, and each is the minimum in this order.
Two tables are equivalent only if their canonic forms are the same.

\tsc{Guru Notes:}
In half of the ten standard enumeration strategies of Sims \cite{Sim}, the
  table is standardised repeatedly.
This is expensive computationally, but can result in fewer cosets being
  necessary.
The effect of doing this can be investigated in \ace\ by (repeatedly)
  halting the enumeration, standardising the table, and continuing; see 
  Section~\ref{ex007} for an example..

\subsection{\ttt{stat[istics]\ /\ stats\ ;}}

If the statistics package is compiled into the code (which it is by
  default, see the \ttt{opt} command), then dump the statistics
  accumulated during the most recent enumeration.
See Section~\ref{ex000} for an example, and the \ttt{enum.c} source file
  for the meaning of the variables.

\subsection{\ttt{style\ ;}}

Prints the current enumeration style, as deduced from the current \ttt{Ct}
  \amp \ttt{Rt} parameters; see Section~\ref{sec:style}.

\subsection{\ttt{subg[roup\ name]\ :\ <string>\ ;}}

This command defines the name by which the current subgroup will be
  identified in any printout.
It has no effect on the actual enumeration, and defaults to \ttt{H}.
An empty name is accepted; to see what the current name is, use the
  \ttt{sr} command.

\subsection{\ttt{sys[tem]\ :\ <string>\ ;}}

Passes \ttt{<string>} to a shell, via the C library routine \ttt{system()}.

\subsection{\ttt{text\ :\ <string>\ ;}}

Just echoes \ttt{<string>}.
This allows the output from a run driven by a script to be tarted up.

\subsection{\ttt{ti[me\ limit]:\ [-1/0/1..]\ ;}}

The \ttt{ti} command puts a time limit (in seconds) on the length of a run.
An argument of $<\!\ttt{0}$ mean there is no limit (the default).
If the argument is $\ge\!\ttt{0}$ then the total elapsed time for this
  call is checked at the end of each pass through the enumerator's main
  loop, and if it's more than the limit the run is stopped and the current
  table returned.
Note that a limit of $\ttt{0}$ performs exactly one pass through the main
  loop, since $\ttt{0} \ge \ttt{0}$.
If the enumerator is run in the continue mode, this allows a form of
  `single-stepping'\kern-1.5pt.
%
The time limit is approximate, in the sense that the enumerator may run for
  a longer, but never a shorter, time.
So, if there is, e.g., a big collapse (so that the time round the loop
  becomes very long), then the run may run over the limit by a large
  amount.

\tsc{Notes:}
The time limit is CPU-time, not wall-time.
As in all timing under Unix, the clock's granularity (usually $10$ mSec)
  and the system load can affect the timing; so the number of main loop
  iterations in a given time may vary.
If you want more precise control, use the \ttt{loop} option.

\subsection{\ttt{tw\ /\ trace[\ word]\ :\ <int>,<word>\ ;}}

Traces \ttt{<word>} through the coset table, starting at coset \ttt{<int>}.
Prints the final coset, if the trace completes.

\subsection{\ttt{wo[rkspace]\ :\ [<int>[k/m/g]]\ ;}}

By default, \ace\ has a physical table size of $10^6$ entries (i.e., 
  $4 \times 10^6$ bytes in the default 32-bit environment).
The number of cosets in the table is the table size divided by the number
  of columns.
The \ttt{wo} command allows the physical table size, in entries, to be
  set.
The argument is multiplied by $1$, $10^3$\kern-2pt, $10^6$\kern-2pt, or 
  $10^9$\kern-2pt, depending as nothing, a \ttt{k}, an \ttt{m}, or a 
  \ttt{g} is appended to the argument.
Although the number of cosets is limited to $2^{31}-1$ (if the C \ttt{<int>}
  type is 32 bits), the table size can exceed the $4$GByte 32-bit limit if
  a suitable machine is used.

\tsc{Notes:}
If the binary option is set (see the \ttt{opt} command), the multipliers
  are $1$, $2^{10}$\kern-2pt, $2^{20}$ \amp $2^{30}$ respectively.
The actual number of cosets in the table is
  $\mathrm{entries}/\mathrm{columns} -2$, rounded down to the nearest
  integer.
The $-2$ is to allow for possible rounding errors and the fact that coset
  \#0 is not used.

\subsection{\ttt{\#\ ...\ <newline>}}

Any input between a sharp sign (\ttt{\#}) and the next newline is ignored.
This allows comments to be included anywhere in command scripts.

%%%%%%%%%%%%%%%%%%%%%%%%%%%%%%%%%%%%%%%%%%%%%%%%%%%%%%%%%%%%%%%%%%%%%%%%%
%%
%E
