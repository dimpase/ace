
% aaa.tex - Colin Ramsay - 2 Feb 01
%
% Abbreviations And Acronyms + technical/(P(E))ACE terms etc.
%
%   5   10   15   20   25   30   35   40   45   50   55   60   65   70   75
% ..|....|....|....|....|....|....|....|....|....|....|....|....|....|....|

This appendix lists:
  the abbreviations and acronyms we use;
  the technical terms we use;
  the various terms used in describing \ace, and in explicating
    its internals;
  any terms specific to \pace\ or \peace\ which are used.
Note that this list includes both terms used in this manual and terms 
  commonly used in the source code.

\begin{tabbing}
xxxxxxxxxx 	\= \kill
\ace       	\> advanced coset enumerator \\ 
aka        	\> also known as \\
alive      	\> an active (non-pending/dead) coset \\
ANSI       	\> American national standards institute \\
arg        	\> argument \\
asap       	\> as soon as possible \\
ave	   	\> average \\
barrier	   	\> sync point at which all threads wait until all are ready \\
\ttt{beg(in)} 	\> starts an enumeration ab initio \\
bn	   	\> between \\
BSD        	\> Berkeley standard distribution \\
C	   	\> the best programming language, \emph{ever} \\
CC         	\> coinc coset processed (enumeration message/phase) \\
CD         	\> coset table definition (enumeration message/phase) \\
cds	   	\> complete definition sequence \\
\ttt{check}	\> synonym for \ttt{redo} \\
Chk        	\> result checking (enumeration message/phase) \\
CL         	\> coset table based lookahead (enumeration message/phase) \\
cmd        	\> command \\
CO         	\> table compaction (enumeration message/phase) \\
coinc      	\> coincidence.  Primary coinc -- occurs during defns/scans. \\
	   	\> Secondary coinc -- consequence of a primary one \\
col	   	\> column \\
concurrent 	\> potentially at the same time, or virtual parallelism \\
\ttt{cont(inue)}\> continues the current enumeration \\
cos  	   	\> coset \\
C(PU)	   	\> aggregated CPU time for an enumeration \\
CPU        	\> central processor unit \\
CRG        	\> the \pace\ style -- coset table, relator tables \amp gap filling \\
CT	   	\> coset table \\
DD         	\> serial deduction stack processing (enumeration message/phase) \\
dead       	\> a fully processed coinc coset \\
dedn       	\> deduction.  Formally -- a deduction made during relator scanning. \\
	   	\> Loosely -- any (new/altered) table entry which is stacked \\
defn       	\> definition \\
defn seq   	\> definition sequence \\
DG         	\> serial gap-filling (enumeration message/phase) \\
\ttt{DOSTK}	\> dedn processing macro, calls appropriate handler \\
DS         	\> definition sequence \\
\ttt{dtime}	\> total elapsed time in \ttt{DOSTK} macro (part of stats) \\
DTT        	\> special debug/test/trace code \\
edp        	\> essentially different position(s) \\
eg	   	\> exempli gratio, for example \\
elt	   	\> element \\
\ttt{end}	\> synonym for \ttt{begin} (don't blame me!) \\
EOF	   	\> end-of-file \\
EOL	   	\> end-of-line \\
Err        	\> error (enumeration message/phase) \\
etc	   	\> et cetera \\
F          	\> FALSE \\
$G$	   	\> the group \\
gen	   	\> generator, either of grp or of subgrp \\
GNU	   	\> GNU's not Unix -- quality `freeware' \\
$g(p)$     	\> growth function of $T$ with $p$ \\
grp	   	\> group \\
$H$	   	\> the subgroup \\
HD         	\> heuristic definition (enumeration message/phase) \\
ie	   	\> id est, that is \\
inc(l)     	\> include/including \\
inc(r)     	\> increase/increasing \\
inv        	\> inverse \\
invol(n)   	\> involution \\
I/O        	\> input/output \\
IP, i/p	   	\> input \\
item       	\> PWs are sequences of items \\
KISS	   	\> keep it simple, stupid \\
\ttt{(kn)h} 	\> coset table rows $<$\ttt{knh} are guaranteed to be complete \\
\ttt{(kn)r} 	\> coset table rows $<$\ttt{knr} are guaranteed to scan at all relators \\
LC	   	\> lower-case \\
len        	\> length \\
lst	   	\> list \\
LWP	   	\> lightweight process -- sorta like a thread, but not quite \\
\ttt{m}, $M$ 	\> MaxCos, the maximum number of cosets active \\
mode	   	\> start, continue or redo an enumeration \\
mutex	   	\> POSIX mutual exclusion lock \\
$n$        	\> the number of slaves/threads (i.e., the argument of \ttt{beg}) \\
n/a	   	\> not applicable \\
\ttt{n(extdf)} 	\> number of next coset to be defined \\
\ttt{nproc}	\> global variable containing value of $n$ \\
NW         	\> non-whitespace (ie, not space, tab, or (maybe) newline \\ 
OP, o/p	   	\> output \\
OS         	\> operating system \\
$p$        	\> the dedn stack batching factor (i.e., the argument of \ttt{pf}) \\
\pace      	\> parallel \ace \\
\tsc{Par}  	\> the parallelisable portion of the running time \\
para       	\> paragraph \\
parallel   	\> actually at the same time, or real parallelism \\
parallel   	\> a \pace\ run with $n \ne 0$ \\
parentheses	\> the ``('' \amp ``)'' characters \\
PC         	\> proof certificate \\
pdl        	\> preferred definition list \\
\peace	   	\> proof extraction after coset enumeration \\
pending    	\> a coset on the coinc queue but not yet processed \\
\ttt{pfactor}	\> global variable containing value of $p$ \\
pthread    	\> POSIX thread \\
PD	   	\> parallel deduction stack processing (enumeration message/phase) \\
PG         	\> parallel gap-filling (enumeration message/phase) \\
pos(n)     	\> position \\
POSIX      	\> portable operating system interface \\
PPP        	\> paranoia prevent problems (ie, belts'n'braces) \\
pri	   	\> primary \\
PT         	\> proof table \\
ptr	   	\> pointer \\
PW	   	\> proof word \\
RD         	\> relator table definition (enumeration message/phase) \\
RA         	\> relator application check (enumeration message/phase) \\
\ttt{redo}	\> redo the current enumeration (keeping the table) \\
red(n)     	\> reduction \\
redundant  	\> a dead coset \\
rel	   	\> relator and/or relation \\
rep        	\> the (current) representative of a coincident coset \\
RS         	\> relators in subgroup (enumeration message/phase) \\
sec	   	\> secondary \\
semaphore  	\> sync primitive allowing signalling between threads \\
seq	   	\> sequence \\
\tsc{Ser}  	\> the serial portion of the running time \\
serial	   	\> a \pace\ run using \ttt{beg:0}, or an \ace\ run \\
SG         	\> subgroup generator (enumeration message/phase) \\
SMP        	\> shared memory multiprocessor and/or symmetric multiprocessing \\
spin-lock  	\> sync via sitting in tight loop until a condition is met \\
square 	   	\> the ``['' \amp ``]'' characters \\
\ brackets 	\> \\
src	   	\> source \\
stats      	\> statistics (package) \\
\ttt{start}	\> synonym for \ttt{begin} \\
strategy   	\> the overall enumeration method (ie, HLT, Felsch, Sims:n, etc) \\
style	   	\> which of the state machines is active (ie, R, C, CR, etc) \\
subgrp	   	\> subgroup \\
\tsc{Sync} 	\> the master-slave synchronisation overhead time \\
sync       	\> synchronous, synchronisation \\
\ttt{t}, $T$	\> TotCos, the total number of cosets defined \\
T          	\> TRUE \\
TAB	   	\> tabulate character \\
TBA	   	\> to be announced/advised \\
thread	   	\> an independent execution sequence within a process \\
tuple      	\> 4-element record of significant scan, see the \ttt{Dlelt} type (in \ttt{al0.h}) \\
UC	   	\> upper-case \\
UH         	\> update hole count check (enumeration message/phase) \\
vs	   	\> versus \\
\tsc{W(all)}	\> elapsed, or wall, time for an enumeration \\
wrd        	\> word \\
WS	   	\> white-space; ie, blanks, tabs, \amp newlines (maybe) 
\end{tabbing}

%TBA ...  
