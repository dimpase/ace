%%%%%%%%%%%%%%%%%%%%%%%%%%%%%%%%%%%%%%%%%%%%%%%%%%%%%%%%%%%%%%%%%%%%%%%%%
%%
%W  al0.tex            ACE standalone documentation          Colin Ramsay
%W                                                            Greg Gamble
%%
%H  $Id$
%%

%%  Colin Ramsay - 11 May 1999  is the true author of this document
%%  Greg Gamble  - 29 Feb 2000 made minor modifications:
%%                * put this header at the top for CVS

%%  The AL0 chapter.
%%
%%  5   10   15   20   25   30   35   40   45   50   55   60   65   70   75
%%..|....|....|....|....|....|....|....|....|....|....|....|....|....|....|

THIS CHAPTER IS OUT OF DATE !!

Level 0 of \ace\ is the actual enumerator.
The routines in Level 0 are the only ones in \ace\ which change the coset
  table and its associated data structure.

\section{Introduction}

\dots concept of a self-consistent table, guaranteed to be the `correct'
  table, or a multiple thereof.

\section{The \ttt{void al0\_enum(void)} function}

The \ttt{al0\_enum()} function assumes that it has been called with a 
  consistent table \amp status, and that its arguments \amp parameters
  are valid.
It is written for speed, not forgiveness; it is up to the caller (e.g.,
  \ace\ Level 1) to do the right thing.

\subsection{Modes, styles \amp phases}

The \ttt{cntrl} argument of \ttt{al0\_enum()} determines which of the 
  three possible \tit{modes} the enumerator will run in.
If $\ttt{cntrl} > 0$ the current table and status are preserved, and
  enumeration continues from where it was interrupted.
This is the \tit{continue} mode and can be used, e.g., if
  the previous enumeration ran out of time.
If $\ttt{cntrl} = 0$ the current table and status are cleared, and a new
  enumeration is performed.
This is the \tit{begin}, \tit{end} or \tit{start} mode, and is that used
  for a new or changed presentation.
If $\ttt{cntrl} < 0$ the current table is preserved, the status is altered
  to indicate the start of the table, and enumeration is commenced at
  coset $1$.
This is the \tit{check} or \tit{redo} mode and can be used, e.g., if 
  relators or subgroup generators were \tit{added} to the presentaion.

The \ttt{rfactor} \amp \ttt{cfactor} parameters determine the
  \tit{style} of the enumeration to be run.
The five valid combinations are listed below.

$\ttt{rfactor}>0$ \amp $\ttt{cfactor}=0$:
The relator table-based style (\tit{R-style}).

$\ttt{rfactor}=0$ \amp $\ttt{cfactor}>0$:
The coset table-based style (\tit{C-style}).

$\ttt{rfactor}<0$ \amp $\ttt{cfactor}>0$:
The coset table-based plus one round of relator table-based style 
  (\tit{Cr-style}).

$\ttt{rfactor}>0$ \amp $\ttt{cfactor}>0$:
The mixed style (\tit{CR-style}), which combines C-style \amp R-style.

$\ttt{rfactor}=0$ \amp $\ttt{cfactor}=0$:
The default style (\tit{Def-style}), which runs R-style and then switches
  to C-style.

The control structure of the enumerator's main loop can be though of as a
  collection of state machines (i.e., finite automata).
The style selects the appropriate \tit{machine}, while the current state
  of a machine is the enumeration's current \tit{phase}.
The mode selects the starting state. 
%
The body of the main loop consists of a collection of blocks of code, each
  of which performs some specific \tit{action} and returns a \tit{result}.
The status and the phase are used to select the next action and phase from
  the machine's transition table.
 
\begin{table}
\hrule
\caption{The modes, styles \amp start states}
\label{tab:msm}
\smallskip
\renewcommand{\arraystretch}{0.85}
\begin{tabular*}{\textwidth}{@{\extracolsep{\fill}}rrrrrr} 
\hline\hline
\ttt{cntrl} & \ttt{rfactor} & \ttt{cfactor} & mode & style & start \\
\hline
$<\!0$ & $<\!0$ & $<\!0$ & redo     & R/C-style & 2  \\
$<\!0$ & $<\!0$ & $0$    & -        & -         & -  \\
$<\!0$ & $<\!0$ & $>\!0$ & redo     & Cr-style  & 1  \\
$<\!0$ & $0$    & $<\!0$ & -        & -         & -  \\
$<\!0$ & $0$    & $0$    & -        & -         & -  \\
$<\!0$ & $0$    & $>\!0$ & redo     & C-style   & 3  \\
$<\!0$ & $>\!0$ & $<\!0$ & redo     & Rc-style  & 30 \\
$<\!0$ & $>\!0$ & $0$    & redo     & R-style   & 4  \\
$<\!0$ & $>\!0$ & $>\!0$ & redo     & CR-style  & 5  \\
\hline
$0$    & $<\!0$ & $<\!0$ & start    & R/C-style & 7  \\
$0$    & $<\!0$ & $0$    & -        & -         & -  \\
$0$    & $<\!0$ & $>\!0$ & start    & Cr-style  & 6  \\
$0$    & $0$    & $<\!0$ & -        & -         & -  \\
$0$    & $0$    & $0$    & -        & -         & -  \\
$0$    & $0$    & $>\!0$ & start    & C-style   & 8  \\
$0$    & $>\!0$ & $<\!0$ & start    & Rc-style  & 35 \\
$0$    & $>\!0$ & $0$    & start    & R-style   & 9  \\
$0$    & $>\!0$ & $>\!0$ & start    & CR-style  & 10 \\
\hline
$>\!0$ & $<\!0$ & $<\!0$ & continue & R/C-style & 12 \\
$>\!0$ & $<\!0$ & $0$    & -        & -         & -  \\
$>\!0$ & $<\!0$ & $>\!0$ & continue & Cr-style  & 11 \\
$>\!0$ & $0$    & $<\!0$ & -        & -         & -  \\
$>\!0$ & $0$    & $0$    & -        & -         & -  \\
$>\!0$ & $0$    & $>\!0$ & continue & C-style   & 13 \\
$>\!0$ & $>\!0$ & $<\!0$ & continue & Rc-style  & 40 \\
$>\!0$ & $>\!0$ & $0$    & continue & R-style   & 14 \\
$>\!0$ & $>\!0$ & $>\!0$ & continue & CR-style  & 15 \\
\hline\hline
\end{tabular*}
\end{table}

\begin{table}
\hrule
\caption{The non-starting states}
\label{tab:phase}
\smallskip
\renewcommand{\arraystretch}{0.85}
\begin{tabular*}{\textwidth}{@{\extracolsep{\fill}}rl} 
\hline\hline
state & meaning \\
\hline
16 & definitions in R-style \\
17 & lookahead in R-style \\
18 & compact in R-style \\
19 & definitions/deductions in C-style \\
20 & lookahead in C-style \\
21 & compact in C-style \\
22 & check finite result \\
23 & C-lookahead in CR-style \\
24 & C-definitions/deductions in CR-style \\
25 & C-compact in CR-style \\
26 & R-definitions in CR-style \\
27 & R-lookahead in CR-style \\
28 & R-compact in CR-style \\
29 & R-definitions in R/C-style \\
\hline\hline
\end{tabular*}
\end{table}

\begin{table}
\hrule
\caption{The actions}
\label{tab:action}
\smallskip
\renewcommand{\arraystretch}{0.85}
\begin{tabular*}{\textwidth}{@{\extracolsep{\fill}}rl} 
\hline\hline
action & meaning \\
\hline
-1 & invalid action (i.e., invalid machine state)\\
0  & null action \\
1  & make some R-style definitions \\
2  & perform lookahead for R-style \\
3  & compact the table \\
4  & make some C-style definitions/deductions \\
5  & perform lookahead for C-style \\
6  & check finite result \\
\hline\hline
\end{tabular*}
\end{table}

\begin{table}
\hrule
\caption{The results}
\label{tab:result}
\smallskip
\renewcommand{\arraystretch}{0.85}
\begin{tabular*}{\textwidth}{@{\extracolsep{\fill}}rll} 
\hline\hline
result & internal & returned \\
\hline
% `normal' terminations/results
$>\!0$  & n/a                             & finite index \\
$0$     & unable to make definition       & overflow \\
$-1$    & successful                      & n/a \\
$-2$    & finite result (potential index) & n/a \\
% `anomalous' terminations
$-256$  & n/a & incomplete table \\
$-257$  & n/a & too many holes \\
$-258$  & n/a & out of time \\
% non-fatal errors
$-512$  & n/a & bad mode/style combination \\
% fatal errors
$-4096$ & n/a & invalid machine state \\
$-4097$ & n/a & invalid finite result \\
\hline\hline
\end{tabular*}
\end{table}

\section{The \ttt{void al0\_compact(void)} function}

\dots can be called even if already compact, so no point in keeping track
  of CT compaction state, just fall through if $\ttt{nextalive} = 
  \ttt{nextdf-1}$.
(There would actually be little penalty in running it on a compacted
  table anyway!)
\dots \tit{don't} call if deductions (up to user to catch this), since
  we discard deductions!
\dots pdl is cleared.
\dots how long does it take, one pass through the table, circa constant
  time?

\dots does \tit{not} produce canonic forms.
Discuss this, and reference illustrative examples.
One serious case where this is useful is for low-index subgroups, where
  one dosn't want duplicates.
Perhaps add a switch that folds standardisation into the compaction
  routine?

\dots table is standardised, so we can find a `small' rep by backtracing
  the rows minimal entries, this does not produce a minimal rep!

\section{Coincidence handling}

The coincidences are stored in the coset table by using the ideas
  described in \cite{Bee}, but not the data structure.
Instead we use the data structure described in \cite{CDHW}.

%%%%%%%%%%%%%%%%%%%%%%%%%%%%%%%%%%%%%%%%%%%%%%%%%%%%%%%%%%%%%%%%%%%%%%%%%
%%
%E
