
% intro.tex - Colin Ramsay - 27 Sep 03, 1 May 14
%
% The introductory chapter.
%
%   5   10   15   20   25   30   35   40   45   50   55   60   65   70   75   80
% ..|....|....|....|....|....|....|....|....|....|....|....|....|....|....|....|

\ace\ is designed to work with partial tables, as well as complete tables
  exhibiting a finite index.
TBA:
Intended user groups ...

\ace\ is divided into three `levels'\kern-1.5pt.
The actual enumerator, called the ``core enumerator''\kern-2pt, is \ace\ 
  Level 0, while the standard driver for the enumerator, the ``core 
  wrapper''\kern-2pt, is \ace\ Level 1.
A stand-alone `example' application, called the ``interactive 
  interface''\kern-2pt, is \ace\ Level 2.
To assist those interested in the actual source code, the function and 
  variable names are prepended with \ttt{AL0\_}\kern+1pt, \ttt{AL1\_} \amp
  \ttt{AL2\_} respectively.
%
\ace\ also includes the ``proof table'' package (PT for short), which can
  be compiled into the executable if required.
The proof table cuts across the level structure, and can only be used as
  part of the interactive interface.
Function and variable names of the PT package are prepended with 
  \ttt{PT\_}\kern+1pt.
TBA: this package ...

TBA: version history, 3.000 vs 3.001, ...
TBA: default build ...

\section{Administrivia}

It is assumed that \ace\ is run on a Unix-box of some description.
TBA: how to compile ...

In order not to clutter-up the body of the text with examples, the bulk of
  these are gathered into a separate appendix.
These examples illustrate many of the features of \ace, and can also serve
  as a source of interesting enumerations.
Some are referred to in the text, but they can all be read independently.
\ace\ script input generating these examples is available in the 
  \ttt{ex***.in} files, as part of the documentation.

\section{Code comments}

$\bullet$
One of the main goals when developing \ace\ 4 was to allow more than the 2G
  cosets allowed by earlier versions.
To do this, we moved to \ttt{long} \ttt{long} types for cosets and various
  counters/integers.
This is not ANSI, but is no real problem since most modern compilers
  support this (and it'll probably be ANSI sometime real-soon).

64 bits is wasteful of memory however (and time too), so we use `packed' 
  data for the coset table entries (typically, we need only go from 4 to 5 
  bytes).
GNU's \ttt{gcc} supports a \ttt{packed} attribute, and is widely available,
  so we used this.
This is very non-ANSI, and so some tinkering may be necessary for other
  compilers (if, indeed, porting is possible).
However, the code is fairly localised, with \ttt{typedef}'s \&
  \ttt{\#define}'s.

One subtletly is that, in some circumstances, we may wish to use 
  \emph{less} than 32 bits for cosets (eg, to save space if tables have few
  rows, but \emph{lots} of columns).
Thus, the \ttt{Coset} type may not be the biggest integer type around, and
  may not be big enough for some purposes.
Further, some integers \emph{may} contain cosets (so need to be as big as
  the biggest \ttt{Coset} type possible) but may contain other things
  (which are bigger than the current \ttt{Coset} type).
So we use \ttt{Coset} types when we're dealing with cosets, \ttt{long} 
  \ttt{long} types when there's a potential size `problem', and \ttt{int} 
  types otherwise.
TBA:\ we should probably define types for all these, to make correcting
  the inevitable mistakes easier!

$\bullet$
The source code is heavily commented, and is considered to be part of the
  documentation.
Conceptually, coset enumeration is straightforward, but there are tricky 
  details and subtle performance issues -- you need to read the source 
  code, to experiment, and to think to appreciate these.

$\bullet$
No attempt is made to cope fancy character codings for the input.
It is \emph{assumed} that the input character stream is a stream of ASCII
  bytes.

\section{References}

The references section of the guide includes various items mentioned in the
  comments in the source code or the example files.
These may not be referred to explicitly in the body of the guide, but they are
  considered part of the documentation and included in the list of references.



